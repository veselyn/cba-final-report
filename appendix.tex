\chapter{Appendix}\label{ch:appendix}

\section{Interview transcripts}\label{sec:interview-transcripts}

\subsection*{Occupation and Teaching Experience}
\begin{itemize}
\item Omar: Occupation?
\item Evangelo: I'm a theater director and a Greek teacher. I teach theater and languages.
\end{itemize}

\subsubsection*{Subjects Taught Online}
\begin{itemize}
\item Omar: What subjects have you taught online?
\item Evangelo: Greek, above all. I’ve done some theater during the pandemic, but it’s difficult to do theater online.
\end{itemize}

\subsubsection*{Motivation for Teaching Online}
\begin{itemize}
\item Omar: What motivates you the most about teaching online?
\item Evangelo: The flexibility to teach from different places and the convenience for both me and the students, as it saves time on commuting.
\end{itemize}

\subsubsection*{Preferred Teaching Mode Online}
\begin{itemize}
\item Omar: What mode do you like to teach online the most? And why?
\item Evangelo: One-to-one is preferred for effectiveness, although group classes are more profitable. Group classes haven’t worked very well for me, though.
\end{itemize}

\subsubsection*{Important Factors for Choosing an Online Platform}
\begin{itemize}
\item Omar: What factors do you consider most important when you use an online platform?
\item Evangelo: Ease of use, cost-effective advertising, and popularity of the platform. For example, SuperProf started well but became expensive for students, leading to fewer contacts.
\end{itemize}

\subsubsection*{Feedback and Ratings}
\begin{itemize}
\item Omar: How important is feedback and ratings to choose a teacher or course?
\item Evangelo: Very important for the initial contact. People look at ratings before reaching out.
\item Omar: And you as a teacher, would you raise the students' ratings?
\item Evangelo: It’s controversial. I’ve had few problems with students. Those who stay, stay. I’m cautious if a student complains about many teachers.
\end{itemize}

\subsubsection*{Participating in Referrals}
\begin{itemize}
\item Omar: Would you participate in referrals to encourage friends to join an online platform?
\item Evangelo: Yes, if the platform is satisfactory and provides benefits like money or free classes. I recommend platforms and services I’m happy with.
\end{itemize}

\subsubsection*{Understanding the Concept of Chronos after introduction to Chronocademy }
\begin{itemize}
\item Omar: “Explained the concept of cashing out with Chronos.”
\item Evangelo: If someone takes my class, they pay with Chronos. At the end of the month, I can choose to cash out or use them for other classes. It’s an exchange system where you accumulate Chronos from teaching and can spend them on learning.
\item Omar: People buy Chronos, creating a pool of money. The platform manages the flow between buying and selling Chronos. Premium features might include no cashout fees and higher profile visibility.
\end{itemize}

\subsubsection*{Market Suitability}
\begin{itemize}
\item Evangelo: Denmark could be a good market because people have more free time and money. They can teach and learn without needing to be professionals.
\end{itemize}

\subsubsection*{Interest in Beta Platform}
\begin{itemize}
\item Omar: Would you be interested in participating in the Beta platform?
\item Evangelo: Yes, especially if there are benefits like premium features and Chronos to start using. I’m open to trying it out, considering how it performs.
\end{itemize}

\subsubsection*{Visibility and Strategy for the Platform}
\begin{itemize}
\item Omar and Evangelo: Visibility is important. A good strategy would be to approach people from platforms like SuperProf. Starting from scratch as an ambassador with privileges could attract interest. We need diverse teachers to ensure it works.
\end{itemize}