\chapter{Appendix}\label{ch:appendix}


\section{Interview guide}\label{sec:interview-guide}

\underline{General Information}
\begin{enumerate}
    \item Occupation:

    \underline{Online Learning Preferences}
    \item Have you taken online courses before?
    \item If yes, what type of courses or subjects?
    \item If not, why?
    What do you dislike about online learning?
    \item What do you like about online learning?
    \item How do you prefer to learn online?

    \begin{itemize}
        \item Live one-on-one classes:
        \item Live group classes:
        \item Pre-recorded video lessons:
        \item Interactive tutorials:
        \item Reading materials and assignments:
        \item Other (please specify):
    \end{itemize}

    \item What factors do you consider most important in online learning and why?
    \begin{itemize}
        \item Instructor's expertise:
        \item Course content and curriculum:
        \item Reviews and ratings:
        \item Price:
        \item Flexibility of schedule:
        \item Duration of the course:
        \item Certification upon completion:
        \item Other (please specify):
    \end{itemize}
    \item Would you be interested in using a time-based online learning platform (e.g., earning time credits by teaching to spend on learning)?

    \underline{Online Teaching Preferences}
    \item What subjects have you taught online?

    \item What motivates you most about teaching online?
    \begin{itemize}
        \item Passion for teaching:
        \item Earning extra income:
        \item Sharing knowledge:
        \item Gaining teaching experience:
        \item Flexibility and convenience:
        \item Earning time credits for learning other skills:
        \item Other (please specify):
    \end{itemize}

    \item How do you prefer to teach online and why?
    \begin{itemize}
        \item Live one-on-one classes:
        \item Live group classes:
        \item Pre-recorded video lessons:
        \item Interactive tutorials:
        \item Reading materials and assignments:
        \item Other (please specify):
    \end{itemize}

    \item What factors do you consider most important when using an online platform like Superprof?
    \begin{itemize}
        \item Ease of use:
        \item Student engagement:
        \item Payment and rewards:
        \item Platform support and resources:
        \item Flexibility of schedule:
        \item Rating and review system:
        \item Integrated video conferencing:
        \item Scheduling and calendar integration:
        \item Progress tracking and analytics:
        \item Community forums and support:
        \item Certification and recognition:
        \item Flexible payment options (money or time credits):
        \item Other (please specify):
    \end{itemize}
    \item Would you be interested in teaching on a time-based online learning platform, rather than for money?

    \underline{Preferences for Platform Features}
    \item How important is feedback and ratings when choosing a teacher or course?
    And for a student?
    \item How likely are you to participate in referrals to encourage friends to join the platform?
    \item What are the main problems or challenges you encounter?

    \underline{Introducing Chronocademy}
    \item Chronocademy is an innovative online platform where you can earn time credits, called Chronos, by teaching skills you know and spend those credits to learn new things.
    No money is needed for this.
    Whether it's languages, math, or programming, you can share your expertise, earn Chronos, and then use those credits to learn from others.
    You can also buy additional Chronos with real money if needed.
    If you prefer to convert your earned Chronos into cash, you can do that too.
    It's flexible, inclusive, and integrated with Google Calendar and Google Meet for smooth organization and video calls.

    \item Would you be interested in participating in the beta version of an online teaching platform like Chronocademy?


    \section{Interview transcripts}\label{sec:interview-transcripts}

    \item \textbf{Interview with Evangelos}

    \subsection*{Occupation and Teaching Experience}
    \begin{itemize}
        \item Juan: Occupation?
        \item Evangelos: I'm a theater director and a Greek teacher.
        I teach theater and languages.
    \end{itemize}

    \subsubsection*{Subjects Taught Online}
    \begin{itemize}
        \item Juan: What subjects have you taught online?
        \item Evangelos: Greek, above all.
        I’ve done some theater during the pandemic, but it’s difficult to do theater online.
    \end{itemize}

    \subsubsection*{Motivation for Teaching Online}
    \begin{itemize}
        \item Juan: What motivates you the most about teaching online?
        \item Evangelos: The flexibility to teach from different places and the convenience for both me and the students, as it saves time on commuting.
    \end{itemize}

    \subsubsection*{Preferred Teaching Mode Online}
    \begin{itemize}
        \item Juan: What mode do you like to teach online the most?
        And why?
        \item Evangelos: One-to-one is preferred for effectiveness, although group classes are more profitable.
        Group classes haven’t worked very well for me, though.
    \end{itemize}

    \subsubsection*{Important Factors for Choosing an Online Platform}
    \begin{itemize}
        \item Juan: What factors do you consider most important when you use an online platform?
        \item Evangelos: Ease of use, cost-effective advertising, and popularity of the platform.
        For example, SuperProf started well but became expensive for students, leading to fewer contacts.
    \end{itemize}

    \subsubsection*{Feedback and Ratings}
    \begin{itemize}
        \item Juan: How important is feedback and ratings to choose a teacher or course?
        \item Evangelos: Very important for the initial contact.
        People look at ratings before reaching out.
        \item Juan: And you as a teacher, would you raise the students' ratings?
        \item Evangelos: It’s controversial.
        I’ve had few problems with students.
        Those who stay, stay.
        I’m cautious if a student complains about many teachers.
    \end{itemize}

    \subsubsection*{Participating in Referrals}
    \begin{itemize}
        \item Juan: Would you participate in referrals to encourage friends to join an online platform?
        \item Evangelos: Yes, if the platform is satisfactory and provides benefits like money or free classes.
        I recommend platforms and services I’m happy with.
    \end{itemize}

    \subsubsection*{Understanding the Concept of Chronos after introduction to Chronocademy}
    \begin{itemize}
        \item Juan: “Explained the concept of cashing out with Chronos.”
        \item Evangelos: If someone takes my class, they pay with Chronos.
        At the end of the month, I can choose to cash out or use them for other classes.
        It’s an exchange system where you accumulate Chronos from teaching and can spend them on learning.
        \item Juan: People buy Chronos, creating a pool of money.
        The platform manages the flow between buying and selling Chronos.
        Premium features might include no cash-out fees and higher profile visibility.
    \end{itemize}

    \subsubsection*{Market Suitability}
    \begin{itemize}
        \item Evangelos: Denmark could be a good market because people have more free time and money.
        They can teach and learn without needing to be professionals.
    \end{itemize}

    \subsubsection*{Interest in Beta Platform}
    \begin{itemize}
        \item Juan: Would you be interested in participating in the Beta platform?
        \item Evangelos: Yes, especially if there are benefits like premium features and Chronos to start using.
        I’m open to trying it out, considering how it performs.
    \end{itemize}

    \subsubsection*{Visibility and Strategy for the Platform}
    \begin{itemize}
        \item Juan and Evangelos: Visibility is important.
        A good strategy would be to approach people from platforms like SuperProf.
        Starting from scratch as an ambassador with privileges could attract interest.
        We need diverse teachers to ensure it works.
    \end{itemize}
    \item \textbf{Interview with Federico}
    \begin{itemize}
        \item \textbf{Juan:} It's recording perfectly, great.
        \item \textbf{Federico:} Exactly, should I respond to you as if it were someone I don't know or what?
        \item \textbf{Juan:} Yes, yes, yes, I mean, as if it were that voice and try to respond as honestly as possible and without your answer being diluted, let's say.
        \item \textbf{Federico:} Alright, perfect.
        \item \textbf{Juan:} Well, sure, I'll give you a brief intro to the product, but without going into too much detail, so you can also get a sense of why I'm asking you these questions.
        \begin{itemize}
            \item The basic idea is to create a digital platform that enables people who want to teach to publish their services, and people who want to learn to find class providers—like an English teacher, guitar teacher, personal trainer, or whatever.
        \end{itemize}
        \item \textbf{Federico:} Perfect.
        \item \textbf{Juan:} So, up to that point, it's like a social network, so to speak, or a marketplace.
        \begin{itemize}
            \item The extra feature is that the platform will give you the option to either pay or charge with real money, but also with a unit we call ``chrono.''
            \item For example, the person receiving a class can pay with ``chronos,'' and the person receiving the ``chronos'' can convert them into real money.
            \item Alternatively, they can keep those ``chronos'' and use them to pay another person for services they want.
        \end{itemize}
        \item \textbf{Federico:} Right.
        \item \textbf{Juan:} Perfect.
        \item \textbf{Juan:} Do I start with a question?
        \item \textbf{Federico:} Yes, go ahead.
        \item \textbf{Federico:} And the ``chrono'' thing, for example, I understand how it works, but how do I obtain a ``chrono''?
        \item \textbf{Juan:} Well, the idea isn’t 100\% defined yet, but the original thought is:
        \begin{itemize}
            \item Every person who joins the platform will receive a welcome gift of a certain number of ``chronos.''
            \item If you give me a class, I pay you with my welcome ``chronos.''
            \item However, you won’t be able to cash them out immediately to prevent financial losses.
        \end{itemize}
        \item \textbf{Federico:} Right.
        \item \textbf{Juan:} So, if you keep those ``chronos,'' you can use them for another class or to boost your position in the catalog—similar to Google Ads.
        \item \textbf{Federico:} Okay.
        \item \textbf{Juan:} As a student, if you run out of ``chronos,'' you can buy more on the platform, and the person receiving them can convert them into real money.
        \item \textbf{Federico:} Perfect.
        \item \textbf{Juan:} So, that's a high-level view of the product.
        \item \textbf{Federico:} Perfect.
        \item \textbf{Juan:} Now, let me pull up a quick schedule.
        I'll start with questions about your preferences for online learning.
        \item \textbf{Juan:} Have you ever taken an online course?
        \item \textbf{Federico:} Yes.
        \item \textbf{Juan:} And was that online course live, face-to-face, or was it pre-recorded?
        \item \textbf{Federico:} Both—live classes and on-demand courses.
        \item \textbf{Juan:} Did you do private classes, like one-on-one English lessons, or was it group sessions?
        \item \textbf{Federico:} Both—private English lessons as well as live group classes.
        \item \textbf{Juan:} Could you briefly mention the skills or topics you wanted to learn?
        \item \textbf{Federico:} Mostly programming, English, and digital marketing.
        \item \textbf{Juan:} And for the on-demand courses?
        \item \textbf{Federico:} Also programming, sales, and project management—more business-related topics.
        \item \textbf{Juan:} What motivates you to choose online learning over in-person learning?
        \item \textbf{Federico:} Convenience—being at home and fitting the courses into my daily schedule.
        For live lessons, private tutors provide better personalization.
        \item \textbf{Juan:} What factors are most important when choosing an online class?
        \item \textbf{Federico:} Reviews are the most important.
        Also, tutor punctuality and professionalism.
        \item \textbf{Juan:} Would you find it useful if the platform saved recordings of your classes for later review?
        \item \textbf{Federico:} Yes, definitely—that would be really helpful.
        \item \textbf{Juan:} Thank you—this feedback is super valuable.
    \end{itemize}
\end{enumerate}
