\chapter{Conclusion}\label{ch:conclusion}

Chronocademy's landing page can be accessed at this \href{https://chronocademy.com}{\underline{link}}.\ Unfortunately, the web app is not hosted on a paid server yet, but the code can be found in this GitHub \href{https://github.com/juaninicolai/chronocademy}{\underline{repository}}, which will be temporarily available as public.

Additionally, the report was written using LaTeX and an automated pipeline which compiles the files into a PDF was implemented.\ This was super helpful and enhanced productivity during the writing phase.\ The LaTeX repository can be found \href{https://github.com/veselyn/cba-final-report}{\underline{here}}

To conclude, the development of Chronocademy was guided by a user-centric approach, leveraging extensive user research and competitive analysis to shape both the user experience (UX) and user interface (UI). The project utilized a robust technology stack, including ESLint, Prettier, TypeScript, Docker Compose, Git, GitHub Actions, Postgres, Kysely, Next.js, TailwindCSS, and ShadCN, to ensure a high-quality, maintainable, and scalable application.

The UX/UI design process was structured around the Double Diamond framework, which facilitated a clear and iterative approach to problem-solving and innovation.\ Key tools such as the Value Proposition Canvas, user personas, and user flows were instrumental in understanding and addressing user needs.\ Wireframes and high-fidelity prototypes enabled rapid ideation and refinement, ensuring a seamless and engaging user experience.

Extreme Programming (XP) principles, including pair programming, ten-minute builds, continuous integration, and micro commits, were integral to the development process.\ These practices enhanced collaboration, code quality, and development efficiency, allowing the team to deliver a robust Minimum Viable Product (MVP) focused on essential features.

Overall, the project successfully combined structured methodologies with hands-on design and development practices, resulting in a platform that aligns with user expectations and provides a cohesive, user-friendly experience.\ The iterative approach and continuous feedback loops ensured that Chronocademy is well-positioned for future growth and enhancement based on real user needs and emerging market trends.