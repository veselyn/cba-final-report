\chapter{Introduction}\label{ch:introduction}

\section{Introduction}\label{sec:introduction}
This project is the result of the combined effort of two ambitious students, Juan and Veselin, as the final step toward completing the Web Development Top-Up program at Copenhagen Business Academy.

The project is based on a business idea developed over the past few months, referred to as Chronocademy.
Its goal is to provide a solution for individuals seeking to learn or teach skills through a centralized platform.
This platform eliminates the need to rely on multiple tools, such as social networks to find collaborators, video conferencing tools for teaching, and financial apps for payments.

This introduction outlines the steps taken to develop the web application, covering product and user experience (UX) perspectives as well as technical decisions and processes.
The development process was guided by tools and theories learned during the program, including agile software development and programming concepts.

The Double Diamond framework was applied for design, the MoSCoW prioritization method was used for feature prioritization, and Extreme Programming (XP) practices guided the development phase.
Details of these approaches are discussed in the following sections.
For software design, the project adopted server-side React patterns and strategies to maintain a unified code style.

The primary technology used in this project was the Next.js framework.
The team focused on writing modern, well-structured, and robust code.
% I think this can be stretched further if we specify what the tools are for,
% i.e., eslint (static code analysis), kysely (query builder with type safety,
% database migrations and seeding), shadcn ui (ui library). Also, I don't know
% if we'll have time for Vitest :D
Modern development practices included implementing continuous integration (CI) and continuous delivery (CD) workflows using GitHub Actions, with deployment managed through the Fly.io hosting service.
The project also utilized the latest ECMAScript features, along with popular tools and libraries such as ESLint, Kysely, and ShadCn UI. Additionally, TypeScript and testing utilities like Vitest were employed to ensure the quality and reliability of the codebase.

Given the framework used, the application follows a monolithic architecture, with a single codebase handling both back-end and front-end aspects.
This approach allows for a single resource to be deployed to the cloud.
The design strongly emphasizes server-side rendered pages to enhance performance, leveraging the benefits of a multi-page application without compromising speed.

\section{Problem Statement}\label{sec:problem-statement}
Challenge: Many people need help learning new skills, especially in technology, but financial barriers prevent access to traditional education.
\newline
\newline
Opportunity: Many individuals possess valuable skills they could teach to others but lack a structured way to exchange them.
\newline
\newline
Solution: Develop a skill exchange platform, Chronocademy, where users teach and learn skills using time credits instead of money.
\newline
\newline
Goal: Evaluate the platform's effectiveness by collecting user feedback on satisfaction, usability, and the perceived value of time-based transactions in education.
